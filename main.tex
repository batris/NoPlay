\documentclass[a4paper,10pt]{article}

\usepackage[utf8]{inputenc}
\usepackage[english]{babel}

\usepackage{xspace,graphicx}
\usepackage{url}
\usepackage{fancyhdr}
\usepackage{textcomp}

\frenchspacing % :-( Alan kind of likes good old English big spaces after full-stops;-)
\setlength{\parindent}{0pt}
\setlength{\parskip}{1ex plus .5ex minus .5ex}

\lhead{DRAFT}
%it would be wonderful to include git versioning in the document, but I have not seen any easy way :-(
\rhead{}
\pagestyle{fancy}

\title{The No-Play Manifesto\footnote{\textbf{Please note that this is so far a very rough working document that only sketches parts of the discussion that will hopefully be expanded upon `real soon now'. Though it may be shown to interested parties, in its current state of development it should not be spread without the authors' explicit concent. }}}
\author{Alan Davidson \& Beatrice Åkerblom\\ 
  \texttt{(alan$|$beatrice)@dsv.su.se}}
\date{\today}


\begin{document}
\maketitle
\tableofcontents

\section{Introduction}


Stockholm University's Department of Computer and Systems Sciences (DSV) has since the 1990s set on a course of first automating the televising of lectures to distant sites, then the recording of lectures, and lately the automated publishing of such lectures for access by students or, if the lecturer so permits, the general public. These recorded lectures are made available through the DSV web subdomain \emph{play} (or nowadays through the second generation version \emph{play2}). In this document we use the term \emph{play} to refer to the system for automatic recording and publishing of lectures. This text represents the beginnings of a discussion that questions whether use of the play system should be regarded as an incontrovertible success.  The goal is to incite and encourage that discussion. 

The DSV teaching staff has been encouraged, yet not required, to make full use of the play facilities. It has lately become a de facto standard to record lectures and students have to a large degree come to expect and plan on the practice. As a consequence students now commonly express dissatisfaction with those staff who do not make recordings. Discussions with such students have indicated that they can be quick to make assumptions about the reasons for not recording, and the primary assumption seems to be that a lecturer wants to force students into lecture halls for fear that they would otherwise suffer from poor self-image when few student attend. Such assumptions can be breed dissatisfaction with staff and their courses and thereby negatively affect the study environment. If the true reasons for questioning the practice of using play are made clear to the students they generally accept them as valid, but having to motivate one's reasons to students is becoming all the more common and time consuming.

Rather than repeating the reasoning behind the decision not to use play to all students who question it, it has been suggested that if the time and trouble were taken to write them down once and for all then some time and effort may ultimately be saved. Therefore this text. Moreover, with this text the authors seek an open discussion on whether the advantages of using play ultimately outweigh the below documented caveats. If that is the case then the authors agree that recording should not be avoided. Conversely, if students and DSV staff find merit to the below arguments it is our hope that further discussion may be a precursor to developing better and more flexible pedagogical tools and environment at DSV.

\section{The Assumed Advantages}
 Even though one may be critical of the play system it clearly has a number of advantages. There has, to our knowledge, not been any formal cost-benefit analysis or evaluation of the play system. We therefore make our own assumptions on what the assumed advantages are:

 \begin{itemize}
   \item Students who have difficulties attending the lectures e.g. through sickness or long travelling times can follow the course.

   \item Students can study the material at their own pace, be it at a slower pace or at a time efficient faster pace.

   \item Fewer students present at lectures can allow lectures to take place in smaller rooms, or in some cases avoid having to hold repeated lectures for large groups. There is therefor a possible economical gain to not needing as large teaching spaces or repeated lecturers.

   \item The material is available for revision purposes prior to summative assessment.

   \item Moving study material to such a digital format is a step in moving to greater audiences than can be an opening to educating greater numbers.
     
   \item The lecturer may be able to use play recordings in developing effective and efficient teaching methods, such as a version of the 'flipped classroom' concept. One undocumented case has come to our attention where a lecturer let students view the recordings made for the previous year, and then spent effort on addressing problems that the students reported. It was reputedly an unpopular strategy with the students, whereas the lecturer reports that this resulted in a significant improvement on examination results.

     
 \end{itemize}
 

%(N.B. all the above reasons can be countered in the following).

 \section{Caveats}

 When questioning the advantages of the play system, the arguments have many dependencies and cross-links. The following headings represent the authors' conception of the most salient points that should be made, though it is understood that the arguments may run into each other.

\subsection{The Assumption of One-Way Communication. }
In the middle ages lectures were ostensibly the reading of a text to a collected audience where one might assume that only one copy of that text was available. Arguably a large proportion of teachers and students have since that time made the assumption that lectures always have been and always will be a one-way communication. 

``Lecturing is that mysterious process by means of which the contents of the note-book of the professor are transferred through the instrument of the fountain pen to the note-book of the student without passing through the mind of either''
(attributed to Edwin E. Slosson in Creative Learning and Teaching by Harry Lloyd Miller, Quote Page 120, Charles Scribner’s Sons, New York. A similar quote often and probably erroneously attributed to Mark Twain, see http://quoteinvestigator.com/2012/08/17/lecture-minds/\#note-4284-1)

Contrast that notion with the idea that the optimum teaching session will be a dialogue between teacher and student. A teacher may be assumed to have a greater knowledge on the studied subject matter, but given that there is no easy method to download that knowledge difference, understanding what the student already understands, what misconceptions they have, what they find difficult to understand, how they learn most effectively, etc. are all things that a good pedagogue will react and adapt to\footnotemark. So an optimum learning session is one-to-one (or why not many pedagogues to one student?), affording maximum interaction. Given the economics of education this situation is untenable, so the good pedagogue will develop methods to attempt to interact as efficiently as possible with larger groups of students. The authors contend that investing in technological solutions that separate the lecturer and the student is misguided when the more urgent issues are how to assist the lecturer in efficiently interacting with increasingly large groups of students.

\footnotetext{There may be one important counterargument here. Perhaps one of the goals of a university education should be for the student to \emph{learn to learn}, i.e. in the course of learning their primary subjects a university student should by the end of their courses be better at learning things on their own that those without a university education. One way to achieve this might be with a ''baptism by fire'', i.e. to subject a student to a diversity of the most difficult learning situations possible, and force them to learn anyway. Forcing a student to become an autodidact might be an ultimate form of pedagogy, that might not require the pedagogue to entire dialogues with the student.}

We believe that the use of the play system is not only based on the idea that a lecture is one-way communication, but unfortunately also promotes that damaging misconception. Since one-way lectures are all too common, when DSV entrenches that unhappy preconception by recording lectures, it is all too understandable that students view lectures as a passive experience rather than an activity that demands heightened concentration. Play limits both teacher and student in ways that we cover in caveats below. Furthermore it is an investment in a method of teaching that the authors believe should be discouraged. Large resources have been ploughed into the development and running of the play system. The authors suggest that if an equivalent budget had been spent on developing better pedagogics, or even in reducing the size of student groups, the results may well have surpassed those we have with the play system in place.


\subsection{Limiting Student Interaction}
To some extent a lecturer will gain a sense of whether the learning session is going well or if they have `lost' their audience. The more the interaction the lecturer promotes from the students the better informed the lecturer on the students' understanding and the lecturer's need to adjust. Modern pedagogical methods promote increasing student interaction at learning sessions. See for example Harvard's \emph{TWENTY WAYS TO MAKE LECTURES MORE PARTICIPATORY}(http://isites.harvard.edu/fs/html/icb.topic58474/TFTlectures.html). The play system disconnects the lecturer from a significant proportion of the students. Moreover one might well fear that it is those students who are most in need of more interaction that are most likely to stay away from lectures.

Some students may be inhibited by the thought that their participation may be included in a recording and therefore choose to be even more passive than they might otherwise. 

As the play system is currently implemented the lecturer must carry a microphone so as to ensure that their voice is included on the recording. There is no easy way for students to have their voices included on the recording, even when asking important questions. What is more, there is no way for the lecturer to use the microphone without the voice also being amplified to the room. This amplification of only one voice in the room, i.e. the lecturer increases the divide between teacher and learner. If one wishes to encourage a sense that students are more equal and active participants then the amplification of one voice is an unfortunate hindrance. Even in large lecture halls a lecturer may be better served by developing the ability to project the voice as in a theatre, rather than relying on amplification, but using play does not permit this.

Interaction with students can be vital for keeping the subject and pedagogics alive for the lecturer. Even as a group, students' prior understanding and misconceptions change over time. Interaction with students constantly challenges lecturers and causes them to develop not only pedagogical methods but also the subject of their expertise. 

According to (find suitable Dylan Wiliam ref) pedagogues are too often occupied with getting through a specified material with a lack of concern about how much is being learned. His analogy it that if teachers were pilots they would set up a flight path based on direction and distance, and once the time available had passed look for a place to land, without much flexibility on re what the real destination was and how prevailing winds have affected the plan. A pedagogue should be able to re-adjust their plan for a lecture based on measures of what the students have understood. If students are not present at lectures, or by the system otherwise discouraged from active participation, then the pedagogue lacks the necessary feedback needed to reevaluate and re-plan as a good pedagogue would.



\subsection{Encourages Poor Study Technique}
When one views the contemporary lecture is seems clear that the number of students who take notes are very few. Since the advent of pre-prepared slide material into lecturing it became a wide assumption that such slides are documentation of a lecture. Students will therefore ask to have slide material published for them beforehand so that they no longer have to spend the lecture time in constantly take notes but can rather concentrate on understanding. To draw this situation to its extreme, the lecturer is expected to provide a complete documentation in the slide material and avoid saying anything outside of the script that the slides do not  Such an extreme situation would make for very limited pedagogics. A learning situation is a process that can and should be supported with all manner of pedagogical and didactic tools and methods.

Having the alternative to review a lecture might be seen as a great advance over documentation-by-slides, meaning that there is even less reason for students to take notes when the whole process of a lecture is documented by video. But this view is (once again) based on the idea that students are passive recipients at lectures. Greater pedagogical gain is surely achieved when the student is instead encouraged to view their task as assimilating the lecture material into their own world view. A very important tool to achieve this is the taking of personal notes. Indeed research indicates that the great advantage to learning comes when students take handwritten notes as opposed to notes on computerised devices.

http://www.scientificamerican.com/article/a-learning-secret-don-t-take-notes-with-a-laptop/

Since play apparently allows students to avoid taking personal notes at lectures the system may be encouraging less effective learning than other methods that would instead encourage students to take notes.


When a student misses a lecture the play system allows them to view it anyway. Consider the alternative. In times when there were no recordings, and even before the common use of presentation slides, students expected to take notes of the whole lecture. A student who would be unavoidably absent from a lecture would have the alternative to follow up on a lecture by asking if they could share their fellow students' notes. It is our firm belief that a situation that encourages students to take lecture notes

\subsection{Might Flexibility Breed Complacency?}
Flexibility in study methods and times would surely be of a great advantage to well motivated students, allowing them to make mature decisions on how to construct their study timetable. But in the current state of university education it seems that only a small minority of students are self-motivated. One might question whether flexibility as always an advantage for the majority of students.

Some students assume that they should be able to prioritise other activities than their university studies and yet still fit their studies in, despite reading campus based full time university studies. Meetings with some students indicate that students expect to be able to use their jobs as a reason for needing special support. Though one may have great sympathy for students who suffer financial difficulties during their studies we contend that students on full time studies must be encouraged to view those studies as a primary priority.

\subsection{The Technology Limits the Pedagogic Stage}
Especially for programming courses, one would have several boards running at the same time.

Since the cameras do not properly follow the lecturer it encourages very static speaking. Using the full breadth of the room to make pedagogical 

Notes made on whiteboards are not legible to the cameras.

The slides and the lecturer are separated. We contend that having a physical connection to the material being discussed is preferable to simply speaking in the background to slide material, or using a disembodied laser point or arrow symbol to indicate individual elements of the slide being discussed. 

\subsection{Insufficient Implementation}
Rife with technical problems that take valuable time from the lecturer's true purpose. 

\subsection{Inefficient Use of the Video Format}
Optimal video learning material is short and to the point. TED talks for example are specifically limited to 18 minutes(https://www.linkedin.com/pulse/20140313205730-5711504-the-science-behind-ted-s-18-minute-rule?forceNoSplash=true). It can include interactive links. Automatically recorded two-hour lectures are a very poor substitute. Given that all of the above mentioned advantages, and some of the caveats, are better served by putting effort into producing good self study materials, including videos, staff should be encouraged to do this more than they are encouraged to allow the automatic systems to do make a poor job of it.

\subsection{Questionable from a Rights and Privacy Viewpoint}
Rights to material and access control.

Students should be asked if they agree to be included. If they are asked in a class they may be unwilling to make their true feelings known for fear of peer pressure. If they were anonymously allowed a flag in Daisy that hindered recording if they were present, would there be any recordings?

Rights of materials shown on the lecture screens (as Mats has had in the case of games).

Sometimes, especially in the security field, a lecturer needs to be able to have some freedom to say things that they would not want recorded. (the question to answer carefully here is 
``why not?''
 to make it clear that the immediate critique that one should never say anything that one cannot stand up for... Taking things recorded out of context is one danger.) 

\subsection{The Power of a Happening}
This section should put forward a case for learning in some situations being enhanced by being in the company of other learners. Some parts are undoubtedly good to do on one's own, but other things happen when in a group, such as:
\begin{itemize}
\item  peer pressure could be positive if one notices that others are more ahead in their understanding. 
\item more chance that others might ask a question that is relevant to your understanding, even when you might not feel comfortable asking yourself.
\item if it is possible to create a positive learning environment then being a part of a learning ``happening'' might enthuse and excite, which in itself can have a very positive influence on learning. Of course the converse may also be true, in that if the learning situation does not manage to excite, but rather the opposite, then being part of a negative happening could be detrimental to learning.
\end{itemize}


Quote the student who said that he was very glad that the lectures were not recorded since if he knew he had the alternative he would sleep in. 


\subsection{Lack of Proper Evaluation}
Students may well appreciate it, but is all they like in their best interests? It should be evaluated for its larger pedagogical effect and value.
Could the vast investment maybe have been better used for other pedagogical initiatives?

\section{Alternatives}
Encourage pedagogics that stretch the concept of what should be achieved and how in two way meetings with students.

Support tools for flipped classrooms.

In situations where two-way meetings are deemed less important or not economically viable, use better prepared formats than recorded lectures that have all the qualities that recorded lectures are assumed to have, and more.


\section{References}

Note, proper referencing should be fixed asap so that any students who happen to see this do not get the wrong idea about how referencing is done (i.e. don't do as I have done so far in the above). /Alan



\end{document} 
