\documentclass[a4paper,10pt]{article}

\usepackage[utf8]{inputenc}
\usepackage[english]{babel}

\usepackage{xspace,graphicx}
\usepackage{url}
\usepackage{fancyhdr}
\usepackage{textcomp}

\frenchspacing % :-( Alan kind of likes good old English big spaces after full-stops;-)
\setlength{\parindent}{0pt}
\setlength{\parskip}{1ex plus .5ex minus .5ex}

\title{The No-Play Manifesto}
\author{Alan Davidson \& Beatrice Åkerblom\\ 
  \texttt{(alan $|$ beatrice)@dsv.su.se}}
\date{\today}

\begin{document}
\maketitle
\tableofcontents

\section{Introduction}
Stockholm University's Department of Computer and Systems Sciences (DSV) has since the 1990s set on a course of first automating the televising of lectures to distant sites, then the recording of lectures, and lately the automated publishing of such lectures for access by students or, if the lecturer so permits, the general public. These recorded lectures are made available through the DSV web subdomain \emph{play} (or nowadays through the second generation version \emph{play2}). In this document we use the term \emph{play} to refer to the system for automatic recording and publishing of lectures. This text represents the beginnings of a discussion that questions whether use of the play system should be regarded as an incontrovertible success.  The title is provocative (and a misnomer, given that it is not really a manifesto) so as to incite and encourage that discussion. 

The DSV teaching staff has been encouraged, yet not required, to make full use of the play facilities. It has lately become a de facto standard to record lectures and students have to a large degree come to expect and plan on the practice. As a consequence students express dissatisfaction with those staff who do not make recordings. Discussions with such students have indicated that they can be quick to make assumptions about the reasons for not recording, and the primary assumption seems to be that a lecturer wants to force students into lectures for fear that they would otherwise suffer from poor self-image when few student attend. Such assumptions can be breed dissatisfaction with staff and their courses and thereby negatively affect the study environment. If the true reasons for questioning the practice of using play are made clear to the students they generally accept them as valid, but having to motivate one's reasons to students is becoming all the more common and time consuming.

Rather than repeating the reasoning behind the decision not to use play to all students who question it, it has been suggested that if the time and trouble were taken to write them down once and for all then some time and effort may ultimately be saved. Therefore this text. Moreover, with this text the the authors seek an open discussion on whether the advantages of using play ultimately outweight the below documented caveats, which would indicate that recording should not be avoided. Conversely, if students and DSV staff find merit to the below arguments it is our hope that further discussion may be a precursor to developing better and more flexible pedagogical tools and environment at DSV. 

\section{The Assumed Advantages}
That students who have difficulties attending the lectures e.g. through sickness or long travelling times can follow the course.

Students can study the material at their own pace, be it at a slower pace or at a time efficient faster pace.

Fewer students present at lectures can allow lectures to take place in smaller rooms, or in some cases avoid having to hold repeated lectures for large groups. There is therefor a possible economical gain to not needing as large teaching spaces or repeated lecturers.

The material is available for revision purposes prior to summative examination.

Moving study material to such a digital format is a step in moving to greater audiences than can be an opening to educating greater numbers.

(N.B. all these reasons can be countered in the following).

\section{Caveats}

\subsection{The Assumption of One Way Communication. }
Hinders development of alternative “modern” pedagogics, flipped classroom.

Students see it an an experience rather than an activity.

Student input not included


\subsection{Limited Audience Feedback}
No sense of whether the audience is following.

Some students may be inhibited by the thought that their participation may be included in a recording and therefore choose to be passive, where active participation would be an advantage for all.

Keeping the subject alive for the lecturer.

According to (find suitable Dylan Wiliam ref) pedagogs are too often occupied with getting through a specified material with a lack of concern about how much is being learned. His analogy it that if teachers were pilots they would set up a flight path based on direction and distance, and once the time available had passed look for a place to land, without much flexibility on re what the real destination was and how prevailing winds have affected the plan. A pedagog should be able to re-adjust their plan for a lecture based on measures of what the students have understood. If students are not present at lectures, or by the system otherwise discoraged from active participation, then the pedagog lacks the necessary feedback needed to reevaluate and replan as a good pedagog would.



\subsection{Encourages Poor Study Technique}
When one views the contempory lecture is seems clear that the number of students who take notes are very few. Since the advent of pre-prepared slide material into lecturing it became a wide assumtion that such slides are documentation of a lecture. Students will therefore ask to have slide material published for them beforehand so that they no longer have to spend the lecture time in constantly take notes but can rather concentrate on understanding. To draw this situation to its extreme, the lecturer is expected to provide a complete documentation in the slide material and avoid saying anything outside of the script that the slides do not  Such an extreme situation would make for very limited pedagocics. A learning situation is a process that can and should be supported with all manner of pedagocical and didactic tools and methods.

Having the alternative to review a lecture might be seen as a great advance over documentation-by-slides, meaning that there is even less reason for students to take notes when the whole process of a lecture is documented by video. But this view is (once again) based on the idea that students are passive recipients at lectures. Greater pedagogical gain is surely acheived when the student is instead encouraged to view their task as assimilating the lecture material into thier own world view. A very important tool to acheive this is the taking of personal notes. Indeed research indicates that the great advantage to learning comes when studnets take handwritten notes as opposed to notes on computerised devices.

http://www.scientificamerican.com/article/a-learning-secret-don-t-take-notes-with-a-laptop/

Since play apparently allows students to avoid taking personal notes at lectures the system may be enouraging less effective learning than other methods that would instead encourage students to take notes.


The ref that says they learn more when taking notes.



When a student misses a lecture the play system allows them to view it anyway. Consider the alternative. In times when there were no recordings, and even before the common use of presentation slides, students expected to take notes of the whole lecture. A student who would be unavoidably absent from a lecture would have the alternative to follow up on a lecture by asking if they could share their fellow students' notes. It is our firm belief that a situation that encourages students to take lecture notes

\subsection{Might Flexibility Breed Complacency?}
Flexibility in study methods and times would surely be of a great advantage to well motivated students, allowing them to make mature decisions on how to construct their study timetable. But in the current state of university education it isseems that only a small minority of students are self-motivated. One might question whether flexibility as always an advantage for the majority of students.

Some students assume that they should be able to prioritise other activities than their university studies and yet still fit their studies in, despite reading campus based full time university studies. Meetings with some students indicate that students expect to be able to use their jobs as a reason for needing special support. Though one may have great sympathy for students who suffer financial difficulties during their studies we contend that students on full time studies must be encouraged to view those studies as a primary priority.

\subsection{The Technology Limits the Pedagogic Stage}
Especially for programming courses, one would have several boards running at the same time.

Since the cameras do not properly follow the lecturer it encourages very static speaking. Using the full breadth of the room to make pedagogical 

Notes made on whiteboards are not legible to the cameras.

The slides and the lecturer are separated. We contend that having a physical connection to the material being discussed is preferable to simple speaking in the background to slide material, or using a disembodied laser point or arrow symbol to indicate individual elements to the slide being discussed. 

\subsection{Insufficient Implementation}
Rife with technical problems that take valuable time from the lecturers true purpose. 

\subsection{Inefficient Use of the Video Format}
Optimal video learning material is short and to the point. It can include interactive links.

\subsection{Questionable from a Rights and Privacy Viewpoint}
Rights to material and access control.

Students should be asked if they agree to be included. If they are asked in a class they may be unwilling to make their true feelings known for fear of peer pressure. If they were anonymously allowed an flag in Daisy that hindered recording if they were present, would there be any recordings?

Rights of materials shown on the lecture screens (as Mats has had in the case of games).

\subsection{The Power of a Happening}
Quote the student who said that he was very glad that the lectures were not recorded since if he knew he had the alternative he would sleep in. 

\subsection{Lack of Proper Evaluation}
Students may well appreciate it, but is all they like in their best interests?It should be evaluated for its larger pedagogical effect and value.
Could the vast investment maybe have been better used for other pedagogical initiatives?

\section{Alternatives}
Encourage pedagogics that stretch the concept of what should be acheived and how in two way meetings with students.

In situations where two way meetings are deemed less important or not economically viable, use better prepared formats than recorded lectures that have all the qualities that recorded lectures are assumed to have, and more.







\end{document} 
